% -----------------------------------------------
% Template for ISMIR 2014
% (based on earlier ISMIR templates)
% -----------------------------------------------

\documentclass{article}
\usepackage{ismir2014,amsmath,cite}
\usepackage{graphicx}
\usepackage{amsfonts}

% Title.
% ------
\title{Music Segmentation Boundaries Analysis: Machines vs Humans}

% Single address
% To use with only one author or several with the same address
% ---------------
%\oneauthor
% {Names should be omitted for double-blind reviewing}
% {Affiliations should be omitted for double-blind reviewing}

% Two addresses
% --------------
%\twoauthors
  %{First author} {School \\ Department}
  %{Second author} {Company \\ Address}

% Three addresses
% --------------
\threeauthors
  {First author} {Affiliation1 \\ {\tt author1@ismir.edu}}
  {Second author} {\bf Retain these fake authors in\\\bf submission to preserve the formatting}
  {Third author} {Affiliation3 \\ {\tt author3@ismir.edu}}

% Four addresses
% --------------
%\fourauthors
%  {First author} {Affiliation1 \\ {\tt author1@ismir.edu}}
%  {Second author}{Affiliation2 \\ {\tt author2@ismir.edu}}
%  {Third author} {Affiliation3 \\ {\tt author3@ismir.edu}}
%  {Fourth author} {Affiliation4 \\ {\tt author4@ismir.edu}}

\begin{document}
%
\maketitle
%
\begin{abstract}
  TODO
\end{abstract}
%
\section{Introduction}\label{sec:introduction}

Organization of the paper.

\section{Machines}\label{sec:eval_desc}

\section{Humans}\label{sec:using_method}

\section{Machines vs Humans}



%\begin{table}
 %\begin{center}
 %\begin{tabular}{|l|l|}
  %\hline
  %String value & Numeric value \\
  %\hline
  %Hello ISMIR  & 2014 \\
  %\hline
 %\end{tabular}
%\end{center}
 %\caption{Table captions should be placed below the table.}
 %\label{tab:example}
%\end{table}

%\begin{figure}
 %\centerline{\framebox{
 %\includegraphics[width=\columnwidth]{figure.png}}}
 %\caption{Figure captions should be placed below the figure.}
 %\label{fig:example}
%\end{figure}

\section{Conclusions}

Our method will save us all.

\begin{thebibliography}{citations}

\bibitem {Author:00}
E. Author:
``The Title of the Conference Paper,''
{\it Proceedings of the International Symposium
on Music Information Retrieval}, pp.~000--111, 2000.

\bibitem{Someone:10}
A. Someone, B. Someone, and C. Someone:
``The Title of the Journal Paper,''
{\it Journal of New Music Research},
Vol.~A, No.~B, pp.~111--222, 2010.

\bibitem{Someone:04} X. Someone and Y. Someone: {\it Title of the Book},
    Editorial Acme, Porto, 2012.

\end{thebibliography}

%\bibliography{ismir2014template}

\end{document}
Instead, we will suppose that $X$ can be hierarchically decomposed into a tree structure.
In this view, we have segmentations at multiple resolutions or levels of granularity.
The question now is how to evaluate with or against tree-structured segmentation.

\subsection{Notation}

Let $S \in \mathcal{P}(X)$ denote a partitioning of the columns of $X$, which can be seen as a flat sequence.
We will use the subscripts $S_R$ and $S_E$ to denote \emph{reference} and \emph{estimation}, respectively.
$S(i)$ will denote the identifier of the partition containing item $i$

Hierarchical segmentations will be denoted by $H$ following the subscript conventions above, and the restriction that $H(i)$ denotes the most specific segment containing $i$.

$H(i, j)$ will denote the least common ancestor of $H(i)$ and $H(j)$.
We will denote precedence (containment) of segments by $\prec$ and $\preceq$ e.g., $H(i) \preceq H(i, j)$.In general, we can interpret $H(\cdot, \cdot)$ as a partial order over pairs: not all pairs are comparable, e.g., if $H(a) = H(b)$.
Restricting to a query frame $q$, $H(q, \cdot)$ induces a partial ranking over the remaining data, in that for all $i, j$.
This will be the key to our evaluation.


\subsection{Reduction to ranking}

Observe that flat segmentations are a special case of hierarchical segmentations, where there is one node at the root of the hierarchy containing all frames, and then $m$ segments at the next level down which partition $X$.

We can reduce segmentation evaluation to a ranking evaluation problem as follows.
Let $q$ denote any frame/column of $X$, and let $i$ and $j$ denote any two frames such that $S_R(q) = S_R(i)$ and $S_R(q) \neq S_R(j)$.
In this case, $i$ may be considered \emph{relevant} for $q$, and $j$ is considered \emph{irrelevant} for $q$.
This leads to a straightforward reduction to bipartite ranking, and induces an evaluation similar to the Area Under the Curve (AUC) statistic. 
Assuming that both $S_E$ and $S_R$ have $n$ frames, we can formally describe the flat segmentation evaluation metric framed under a ranking problem as:

\begin{multline}
f(q ; S_E) := \frac{1}{|S_R(q)|\cdot (n - |S_R(q)|)} \sum_{i \in S_R(q)} \sum_{j \notin S_R(q)} \\ \mathbb{I}\left[ S_E(q) = S_E(i) \neq S_E(j) \right]
\end{multline}

That is, the score for frame $q$ is the fraction of pairs $(i, j)$ for which $S_E$ agrees with $S_R$ with respect to $q$ (i.e. membership in the segment).

Averaging over all $q$ gives the AUC score:

\begin{equation}
AUC(S_E) := \frac{1}{n} \sum_q f(q ; S_E)
\end{equation}

\subsection{Partial Ranking}

Note that $f$ is defined in terms of membership (in)equalities.
In the flat case, we can equivalently reason about strict precedences.
Rather than compare $i$ and $j$ where $S(q) = S(i) \neq S(j)$, we can express this as $H(q, i) \prec H(q, j)$.
That is, $q$ and $i$ merge before $q$ and $j$:

\begin{equation}
g(q ; H_E) := \frac{1}{Z_q} \sum_{i,j : H_R(q, i) \prec H_R(q, j)}  \mathbb{I}\left[ H_E(q, i) \prec H_E(q, j) \right]
\end{equation}

where $Z_q$ is a normalization term that counts the number of terms in the summation.

$g$ can be viewed as a Generalized AUC (GAUC) over the partial ranking induced by the hierarchical segmentation.
Note that the precedence comparisons here are strict, so we never compare two $i$ and $j$ that merge simultaneously with $q$.

Aggregating over all frames $q$, we get the following:

\begin{equation}
GAUC(H_E) := \frac{1}{n} \sum_q g(q ; S_E)
\end{equation}

Note, this loss is equivalent to that evaluated by (McFee \& Lanckriet, 2011) for subjective artist similarity.
The difference here is that the reference rankings are induced from ordinal data, and not subject to consistency errors.

\subsection{Windowing in Time}

The computation of GAUC can be expensive ($O(n^3)$ using a straightforward implementation).
The score will also vary depending on $n$, so it would be hard to make intra tracks comparisons.
Therefore, we propose to use a time window of $t$ seconds in order to both simplify the computation of the metric and normalize its dynamic range.

\section{Using GAUC}\label{sec:using_method}

In this section we discuss the behavior of GAUC by showing various synthetic examples, and compare them against other existing methods when possible.
We subdivide this section based on the type of annotations to be evaluated.

\subsection{Choosing a Time Window}

There's a time frame 


\subsection{Flat vs Flat}



\subsection{Flat vs Hierarchical}

\subsection{Hierarchical vs Hierarchical}

\section{Evaluating Automatic Algorithm}

Olda with SALAMI.


%\begin{table}
 %\begin{center}
 %\begin{tabular}{|l|l|}
  %\hline
  %String value & Numeric value \\
  %\hline
  %Hello ISMIR  & 2014 \\
  %\hline
 %\end{tabular}
%\end{center}
 %\caption{Table captions should be placed below the table.}
 %\label{tab:example}
%\end{table}

%\begin{figure}
 %\centerline{\framebox{
 %\includegraphics[width=\columnwidth]{figure.png}}}
 %\caption{Figure captions should be placed below the figure.}
 %\label{fig:example}
%\end{figure}

\section{Conclusions}

Our method will save us all.

\begin{thebibliography}{citations}

\bibitem {Author:00}
E. Author:
``The Title of the Conference Paper,''
{\it Proceedings of the International Symposium
on Music Information Retrieval}, pp.~000--111, 2000.

\bibitem{Someone:10}
A. Someone, B. Someone, and C. Someone:
``The Title of the Journal Paper,''
{\it Journal of New Music Research},
Vol.~A, No.~B, pp.~111--222, 2010.

\bibitem{Someone:04} X. Someone and Y. Someone: {\it Title of the Book},
    Editorial Acme, Porto, 2012.

\end{thebibliography}

%\bibliography{ismir2014template}

\end{document}
